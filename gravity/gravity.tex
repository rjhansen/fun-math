\documentclass[10pt,letterpaper]{report}
\usepackage{mathtools}
\usepackage{libertine}
\usepackage{eulervm}
\usepackage{longtable}
\usepackage{hyperref}
\hypersetup{
    colorlinks = true,
    allcolors = blue
}
\usepackage{url}
\begin{document}
\title{Newtonian Gravity}
\author{Robert J.\ Hansen\thanks{\href{mailto:rob@hansen.engineering}{\nolinkurl{rob@hansen.engineering}}}}
\maketitle
\begin{abstract}

  Isaac Newton's law of universal gravitation sounds complicated, but in
  reality you can do surprising things armed with it and a little high
  school algebra.  Here we're going to use it to figure out, ``So
  where between the earth and the moon would their two gravities
  cancel out?''

  The latest version of this document can be found at
  \href{https://www.github.com/rjhansen/fun-math}{GitHub}.

  Copyright 2017, Robert J.\ Hansen.  You may share and adapt this
  work under the terms of the
  \href{https://creativecommons.org/licenses/by-sa/4.0/}{Creative
    Commons Attribution-ShareAlike 4.0 International license}.
  
\end{abstract}
\tableofcontents
\part{Student Section}
\chapter{Newton's Gravity}
\section{Introduction}
Imagine you wanted to send a baseball all the way to the moon.  How
far would you have to hit it?

The moon is about 384,000 kilometers away, but you wouldn't need to
hit it that far.  Somewhere between the earth and the moon, there's a
\textit{cancellation point} where the earth's gravity and the moon's
gravity cancel out.  If you hit the ball even one meter short of this
cancellation point it will fall back to earth --- but if you hit one
meter beyond, the moon's gravity will draw you the rest of the way
there.

So where is the cancellation point?  How far would we have to hit a
baseball so the moon's gravity could take it the rest of the way?
With a little algebra and Newton's theory of gravity, we can figure it
out.

\subsection{The good news}
\textit{Don't be scared by this document's size!}  Yes, it's very
large, but that's only because each and every step is fully
documented.  Many steps which are glossed over in other treatments are
presented here fully.  We'll conquer this problem by taking a large
number of very tiny steps in the direction of an answer, and each step
will be fully explained.

\subsection{Intended audience}
Anyone interested in mathematics, physics, astronomy, or space
travel.

\subsection{Required skills}
In order to make sense of this you'll need to have a good handle on
basic algebra using one variable, square roots, and exponents.  This
won't teach you any algebra, but it will show you how you can use
algebra to answer interesting questions about the universe.

\subsection{You have permission}
This entire thing is a workbook---so work in it!  Underline things, take
notes, circle stuff.  Real scientists mark up their workbooks.  They
do it because it helps them understand.  You should feel free to do
the same.

\section{Newton's Law of Universal Gravitation}

In 1687, Isaac Newton published a groundbreaking science book called
\textit{Philosophi\ae{} Naturalis Principia
    Mathematica}.  Like all science books of his day, he wrote it in
Latin.  Its name means, ``The Mathematical Principles of
Physics''.  (Back then, physics was called ``natural
  philosophy''.)  Most people just call his book ``the
  \textit{Principia}''.

In it, Newton put forth the following idea about how gravity works.
``The force of gravity between two objects is equal to what
you get by multiplying a certain constant with the masses of the two
objects, and dividing that by the square of how far apart they
are.''

By long-standing custom, constants (things that don't change in an
equation) receive capital letters.  Variables (things which are
allowed to change) receive lower-case letters.  For right now
everything is going to be fixed, so you'll see a lot of capital
letters.  Rewriting Newton's idea in mathematical-ese, we get:

\[
F = G\frac{M_1 M_2}{D^2}
\]

There are other ways this could be written.  You may sometimes see
this as the equivalent,

\[
F = \frac{G M_1 M_2}{D^2}
\]

They mean the same thing: ``take the gravitational constant, multiply
it by the first mass, multiply that by the second mass, and divide
everything by the square of the distance between them.''

\subsection{The gravitational constant}
It took many scientists many years to figure out the value of $G$ in
this equation.  Today our best estimate is that $G = 6.67408 \times
10^{-11}$.  That number is such a mouthful to say that most physicists
will just call it $G$ and let that letter stand in for it all.  (This
is sort of like letting ``the \textit{Principia}'' stand in for the
actual name of Newton's book, just in math-ese.)

\subsection{An example}
Imagine you have a brick and you want to know how much it
weighs. You'll need to know the mass of the earth and the mass of the
brick, as well as how far you are from the earth's center.  The
earth's mass ($M_1$) is $5.972 \times 10^{24}$ kilograms, and standing
on the earth you're about $6.371 \times 10^6$ meters from its center
($D$).  The brick's mass ($M_2$) is $1$ kilogram.

How will we proceed?

\begin{longtable}{| c | p{.45\textwidth} | p{.45\textwidth} |}
  \hline
  Step & Formula & Explanation \\
  \hline \hline
  1 & \[ \overbrace{6.67408 \times 10^{-11}}^G \times
    \frac{\overbrace{5.972 \times 10^{24}}^{M_1}\times
      \overbrace{1}^{M_2}}{\underbrace{(6.371 \times
        10^6)^2}_{D^2}} \] & Start by writing out the
  original equation.  You can use curly brackets to make notes to
  remind you of what means what, if that helps you. \\
  \hline
  2 & \[ \overbrace{6.67408 \times 10^{-11}}^G \times
  \frac{\overbrace{5.972 \times 10^{24}}^{M_1}}{\underbrace{(6.371
      \times 10^6)^2}_{D^2}} \] & Any time we multiply by 1, just remove it.
  Multiplying by 1 never changes anything. \\
  \hline
  3 & \[
    G \times \frac{M_1}{D^2} = \frac{GM_1}{D^2} \]
  & Any fraction
  $\frac{B}{C}$ can be thought of as $B \div C$.  So, 
  $A \times \frac{B}{C}$ is the same as $A \times B \div C$.
  Thanks to the Associative Property of Multiplication, this is the same as
  $(A \times B) \div C$, or $\frac{A \times B}{C}$, or $\frac{AB}{C}$.\\
  \hline
 4 & \[ \frac{\overbrace{6.67408 \times 10^{-11}}^{G}\times\overbrace{5.972
     \times 10^{24}}^{M_{1}}}{\underbrace{(6.371 \times
     10^6)^2}_{D^2}}\] & Rewrite $\frac{GM_1}{D^2}$ using our actual
 numbers \\
 \hline
  5 & \[ \frac{6.67408 \times 5.972 \times 10^{-11} \times
    10^{24}}{(6.371 \times 10^6)^2} \] & $ A \times B \times C \times D$
  can be rearranged into $A \times C \times B \times D$ thanks to the
  Commutative Property of Multiplication.  So long as we don't remove
  any numbers from our string of multiplicands, we can rearrange them
  how we like.  In just a little bit this will make life a lot easier!
  \\
  \hline
  6 & \[ \frac{6.67408 \times 5.972 \times 10^{13}}{(6.371\times10^6)^2}\] &
  Two exponential numbers with the same base can be
  easily multiplied together just by adding their exponents.
  $10^{-11}$ and $10^{24}$ both have the same base ($10$).
  $-11 + 24 = 24 - 11 = 13$, so $10^{-11} \times 10^{24} = 10^{13}$.
  Now, isn't that a lot nicer? \\
  \hline
  7 & \[ 6.67408 \times 5.972 \approx 39.86 \] & We don't need all the
  digits here: $39.86$ is plenty accurate for us.  (Scientists have a
  special procedure for determining how many digits to use, called
  ``determination of significant digits''.  We won't get into it here:
  for right now, just assume two digits past the decimal point is
  pretty
  good.) \\
  \hline
  8 & \[ \frac{39.86 \times 10^{13}}{(6.371 \times
    10^6)^2} \] & Another two unpleasant numbers are gone, replaced
  with a single much nicer number \\
  \hline
  9 & \[ 39.86 = 3.986 \times 10^1 \] & When using scientific notation,
  any number larger than ten should be broken up into a number smaller
  than ten multiplied by a certain power-of-ten.  Any number smaller
  than one should be changed in a similar way. \\
  \hline
  10 & \[ \frac{3.986 \times 10^1 \times 10^{13}}{(6.371 \times
    10^6)^2} \] & Re-write step 8 with the outcome of step 9 \\
  \hline
  11 & \[ \frac{3.986 \times 10^{14}}{(6.371 \times 10^6)^2} \] &
  The numerator is now complete, and less scary. \\
  \hline
  12 & \[ \frac{3.986 \times 10^{14}}{6.371^2 \times {10^6}^2} \] &
  $(a \times b)^2$ is the same as $a \times b \times a \times b$.
  Thanks to the Commutative Property of Multiplication, this is the
  same as $a \times a \times b \times b$, or $a^2 \times b^2$. \\
  \hline
  13 & \[ 6.371 \times 6.371 \approx 40.59 \approx 4.059 \times
  10^1 \] & Multiply out the first squared term and convert it into a
  proper scientific-notation number \\
  \hline
  14 & \[ \frac{3.986 \times 10^{14}}{4.059 \times 10^1 \times
    {10^6}^2} \] & Re-write step 12 with the outcome of step 13 \\
  \hline
  15 & \[{10^6}^2 = 10^{12}\] & When raising a power to a power,
  multiply together the two powers.  E.g., $100^{3} = 1000000$,
  or ${10^2}^3 = 10^6$. \\
  \hline
  16 & \[\frac{3.986 \times 10^{14}}{4.059 \times 10^1 \times
    10^{12}}\] & Re-write step 14 with the outcome of step 15 \\
  \hline
  17 & \[ 10^1 \times 10^{12} = 10^{13}\] &
  Following our rule from before (``to multiply together two
  exponential numbers in the same base, add their exponents''), we
  simplify the denominator. \\
  \hline
  18 & \[\frac{3.986 \times 10^{14}}{4.059 \times 10^{13}} \] &
  Re-write step 16 with the outcome of step 17 \\
  \hline
  19 & \[\frac{3.986 \times 10^1}{4.059}\] & Both numerator and
  denominator share a factor of $10^{13}$.  It cancels out, leaving
  $10^1$ on top and nothing below. \\
  \hline
  20 & \[\frac{3.986 \times 10}{4.059}\] & $10^1 = 10$, so simplify
  step 19 with that \\
  \hline
  21 & \[10 \times \frac{3.986}{4.059}\] & Sort of like step 3, but in
  reverse \\
  \hline
  22 & \[3.986 \div 4.059 \approx 0.982\] & Plain-old long division \\
  \hline
  23 & \[10 \times 0.982 = 9.82 \] & Re-write step 21 with step 22 \\
  \hline
  24 & \[9.82\] & {\Huge{\textbf{Success!}}} (Huge boldface caps?  Oh, yes,
  you deserve it!) \\
  \hline   
\caption{A full workup of using Newton's Law of Universal Gravitation}  
\end{longtable}

At the end of all this math, you've discovered that at the earth's
surface the earth and a 1-kilogram brick pull on each other with a
force of $9.82$ somethings.

Remember: kilograms measure mass (how much of something there is),
\textit{not weight.}  Weight is a measure of how hard something is
being pulled down.  If you were to stand on the moon your mass would
remain the same (none of you would be missing), but your weight would
be significantly less (you'd feel lighter).

So if the kilogram isn't a unit of weight, what is?

Scientists use the kilogram as the unit of mass, and the
\textit{newton} as the unit of weight.  (You'll also see the newton
called the unit of force.  It's the same thing, really: weight is just
``the force with which gravity is pulling you down''.)

So, at the earth's surface, you've shown a 1-kilogram block weighs
$9.82N$.  The capital N is used as the abbreviation for newton, much
in the same way ``lb.'' is used for pound, or ``kg'' is used for kilogram.

\subsection{Homework questions}
\begin{enumerate}
\item
If you wanted to find out how hard gravity would pull down a 1-kg brick on
the moon, what would you need to know about the moon before you could start
solving the problem?
\item
So how hard should a 1-kilogram brick be pulled down on the surface of
the moon?  (You can use Google to find the numbers you need.)
\item
Repeat it for Mars.  Which of the three (the earth, the moon, and Mars) has the
strongest gravity at their surface?  The moon's gravity is about what
fraction of the earth's?  The gravity of Mars is about what fraction of
the earth's?
\end{enumerate}

\section{Finding where gravity cancels out}

To find where gravity cancels out we need to find where the force of
earth's gravity is exactly countered by the force of the moon's
gravity.  Once more, let's break out our trusty 1-kilogram brick (a
useful companion in almost any physics experiment!) and figure out where the forces on it are equal.

To help keep things straight (``is that the force from the earth, or from the
moon?'') we'll use two special symbols: $e$ will denote the earth, $o$
will denote whatever object we're looking to balance between the earth
and the moon, and $m$ will denote the moon.  $M_{e}$ would be, for
instance, ``the mass of the earth'', and $M_{o}$ would be,
``the mass of the object between the earth and the moon''.

We'll let $D$ be the total distance between the center of the earth
and the center of the moon.  If the brick is $x$ meters from earth, it
will be $D-x$ meters from the moon.

First, we write out the formula for the force on the brick from the earth and
the moon:

\newpage

\begin{longtable}{| c | p{0.45\textwidth} | p{0.45\textwidth} |}
  \hline
  Step & Formula & Explanation \\
  \hline \hline
  1 & \[ G\frac{M_e M_o}{x^2} = G\frac{M_m M_o}{(D - x)^2} \] & The
    force of Earth's gravity on the object $x$ meters away from the
    Earth's center equals the force of the moon's gravity on the
    object $D - x$ meters away from the moon's center\\
    \hline
  2 & \[ GM_o \frac{M_e}{x^2} = GM_o\frac{M_m}{(D - x)^2} \] & Use the
  Associative Property of Multiplication to move our $M_o$ term out
  front with the $G$ term \\
  \hline
  3 & \[ \frac{M_e}{x^2} = \frac{M_m}{(D - x)^2} \] & Divide both
  sides of the equation by $GM_o$ \\
  \hline
  4 & \[ \frac{\sqrt{M_e}}{x} = \frac{\sqrt{M_m}}{D - x} \] & Run a
  square root over both sides \\
  \hline
  5 & \[ \frac{(D-x)\sqrt{M_e}}{x} = \sqrt{M_m} \] & Multiply
  everything through by $D - x$ \\
  \hline
  6 & \[ \frac{D-x}{x} = \frac{\sqrt{M_m}}{\sqrt{M_e}} \] & Divide
  everything through by $\sqrt{M_e}$ \\
  \hline
  7 & \[ \frac{D - x}{x} = \sqrt{\frac{M_m}{M_e}} \] & Normalize the
  fraction on the right \\
  \hline
  8 & \[ \frac{D}{x} - \frac{x}{x} = \sqrt{\frac{M_m}{M_e}} \] &
  Separate the left-hand side into two fractions \\
  \hline
  9 & \[ \frac{D}{x} - 1 = \sqrt{\frac{M_m}{M_e}} \] & $\frac{x}{x} =
  1$ \\
  \hline
  10 & \[ \frac{D}{x} = 1 + \sqrt{\frac{M_m}{M_e}} \] & Add $1$ to
  both sides \\
  \hline
  11 & \[ D = x\left( 1 + \sqrt{\frac{M_m}{M_e}}\right) \] & Multiply both sides by
  $x$ \\
  \hline
  12 & \[ \frac{D}{1 + \sqrt{\frac{M_m}{M_e}}} = x \] & Divide both
  sides by $1 + \sqrt{\frac{M_m}{M_e}}$\\
  \hline
  13 & \[ x = \frac{D}{1 + \sqrt{\frac{M_m}{M_e}}} \] & Flip sides \\
  \hline
\caption{Using the Law of Universal Gravitation to derive a cancellation
  point}
\end{longtable}

We've now determined that for \textit{any} two bodies of mass, we
can find a cancellation point between them by dividing the distance
between them by one plus the square root of their mass ratio!

$M_m$ is about $7.35 \times 10^{22}$ kilograms;
$M_e$ is about $5.972 \times 10^{24}$ kilograms;
and they're separated by about $3.844 \times 10^8$ meters.

\begin{longtable}{| c | p{0.45\textwidth} | p{0.45\textwidth} |}
  \hline
  Step & Formula & Explanation \\
  \hline \hline
  1 & \[ \frac{3.844 \times 10^8}{1 + \sqrt{\frac{7.35 \times
        10^{22}}{5.972 \times 10^{24}}}} \] & Re-write our derived
    equation, this time with our real numbers \\
    \hline
  2 & \[ \sqrt{\frac{7.35 \times 10^{22}}{5.972 \times 10^{24}}} \] & Move
  that awful fraction out so we can wrestle with it by itself \\
  \hline
  3 & \[ \sqrt{\frac{7.35}{5.972 \times 10^{2}}}\] & Cancel out a common factor
    of $10^{22}$: already it looks a lot nicer! \\
  \hline
  4 & \[ \sqrt{\frac{7.35}{5.972} \times \frac{1}{10^2}} \] & Separate it out
  \\
  \hline
  5 & \[ \sqrt{\frac{7.35}{5.972}} \times \sqrt{\frac{1}{10^2}} \] & 
  Separate it into two square root terms \\
  \hline
  6 & \[ \sqrt{\frac{7.35}{5.972}} \times \frac{1}{10} \] & Simplify the
  second square root \\
  \hline
  7 & \[ 0.1 \times \sqrt{\frac{7.35}{5.972}} \] & Reorganize, using
  the fact $0.1 = \frac{1}{10}$ and the Commutative Property of
  Multiplication \\
  \hline
  8 & \[ 7.35 \div 5.972 \approx 1.231 \] & long division to the
  rescue \\
  \hline
  9 & \[ 0.1 \times \sqrt{1.231} \] & rewrite step 7 with step 8 \\
  \hline
  10 & \[ 0.1 \times 1.11 \] & I admit, I used a pocket calculator
  here; 1.11 is an approximate value\\
  \hline
  11 & \[ 0.1 \times 1.11 = 0.111 \] & complete step 10 \\
  \hline
  12 & \[ \sqrt{\frac{7.35 \times 10^{22}}{5.972 \times 10^{24}}}
  \approx 0.111 \] & Equate step 2 with the outcome of step 11 \\
  \hline
  13 & \[ \frac{3.844 \times 10^8}{1 + 0.111} \] & Re-write step 1 \\
  \hline
  14 & \[ \frac{3.844 \times 10^8}{1.111} \] & Simplify the
  denominator \\
  \hline
  15 & \[ 10^8 \times \frac{3.844}{1.111} \] & Wow.  Isn't this a
  \textit{much} nicer fraction? \\
  \hline
  16 & \[ 10^8 \times 3.46 \] & Long division to the rescue once again
  \\
  \hline
  17 & \[ 3.46 \times 10^8 \] & The distance to the gravity cancellation
  point, in meters \\
  \hline
  \caption{Computing a gravity cancellation point}  
\end{longtable}

We've found a gravity cancellation point $3.46 \times 10^8$ meters from the
  earth, or about ninety percent of the way to the moon.  If we can
  just get there, the moon's gravity will take over and draw us the
  rest of the way in.

  \subsection{Homework problems}
  \begin{enumerate}
  \item
    How far is $3.46 \times 10^8$ meters, in kilometers?
  \item
    How far is $3.46 \times 10^8$ meters, in miles?  (Hint: there are
    about $1.61$ kilometers per mile.)
  \item
    There are five lunar libration points.  These are also called
    ``Lagrange points'', after the astronomer who discovered them,
    Joseph-Louis Lagrange.  They're named L1, L2, L3, L4, and L5.  If
    something is positioned exactly on a Lagrange point, it will be
    effectively immobile relative to the earth and moon.  This is very
    much like a gravity cancellation point, but it's not quite identical.

    Look up where the L1 point is.  You'll discover that our gravity
    cancellation point is almost, \textbf{but not quite}, the same as
    L1.
    There must be something
    else going on with the earth and the moon that's affecting our
    numbers and throwing them off.  What could it be?
  \item
    How is a Lagrange point different from a gravity cancellation
    point?
  \item
    Computing a Lagrange point accurately requires solving a
    \textit{quintic equation}.  What's a quintic equation?  Is there
    any general method to solve one?
  \end{enumerate}

  \part{Teachers' and Parents' Section}
  \chapter{Answers to Homework Questions}

  This section is meant for teachers and parents, and for students
  \textit{after} they've tried to solve the homework problems on their
  own.
  
  \section{\S 1.2.3}

  \begin{enumerate}
  \item
    The moon's mass in kilograms ($7.35 \times 10^{22}$), and the
    moon's radius in meters ($1.737 \times 10^6$).
  \item
    Repeating the same process we used to compute the earth's gravity,
    we discover a $1$-kilogram brick on the moon would be pulled down
    with a force of $1.63N$.  This is almost exactly one-sixth the
    force felt on the earth.
  \item
    Mars's mass is $6.42 \times 10^{23}$ kilograms; its radius, $3.39
    \times 10^6$ meters.  This means a $1$-kilogram brick on its
    surface weighs $3.73N$, or about two-fifths what it weighs on earth's
    surface.
  \end{enumerate}

  \section{\S 1.3.1}

  \begin{enumerate}
  \item
    To convert from meters to kilometers, divide by $1000$.  You can
    do this easily by seeing that $1000 = 10^3$.  Then just use the
    subtract-the-exponents trick to do the division: $3.46 \times 10^8
    \div 10^3 = 3.46 \times 10^5$ kilometers, or 346,000 kilometers.
  \item
    $3.46 \times 10^5 \div 1.61 = 10^5 \times \frac{3.46}{1.61}$.
    That fraction is about $2.15$, so the cancellation point is about
    215,000 miles away.
  \item
    Lagrange points aren't the same as gravity cancellation points,
    but they're similar.  Lagrange points are misnamed: they're
    actually Lagrange \textit{orbits}.  Something at one of the
    earth's Lagrange points is still traveling in a
    circle (an ellipse, really) around the earth; it's just that at
    each step along its orbit it preserves the exact same distances
    between itself, the earth, and the moon.

    In our derivation we assumed the earth and the moon were fixed in
    space.  In reality, though, the moon is in constant motion around
    the earth.  The Lagrange points represent cancellation in a moving
    system.  We computed cancellation points in an unmoving
    \mbox{system ---} what physicists would call a \textit{static}
    system.

    \textbf{This doesn't mean we're wrong!}  Quite the opposite.  We
    correctly computed the gravity cancellation point.  If you can hit
    a baseball (or launch a rocket) to that point, the moon's gravity
    \textbf{will} take it the rest of the way.

    All it means is that we're computing a straight-line journey
    between the earth and the moon.  A Lagrange point is a stable
    orbit around the earth, which is to say it's a circle around the
    earth without a beginning or ending.

    Lagrange points are closely related to cancellation points, but
    they're also quite distinctly different.
  \item
    See previous answer.
  \item
    A quintic equation is one where there's a fifth-power variable.
    Even simple quintic equations, such as $x^5 - x + 1 = 0$, can be
    murderously hard to solve with algebra.  One of the major
    motivating forces behind the invention of calculus was the need
    for a more powerful kind of math that could address problems like
    these.
  \end{enumerate}
\end{document}
